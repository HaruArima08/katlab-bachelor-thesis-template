\chapter{\tool{}の機能}\label{cha:Function}

本章では、本研究で試作するツール\tool{}について説明する。\\
以降、\tool{}の入出力、初期設定、および、\tool{}使用時の流れについて説明する。

% 入出力
\section{入出力}


% 初期設定
\section{初期設定}
ツール使用者は\tool{}を実行するために、プロジェクトの情報を、\tool{}の設定ファイルvitt.yamlに事前に記述する必要がある。
ツール使用者が設定ファイルvitt.yamlに記述する必要がある情報を、以下に示す。

\begin{itemize}
    \item システム名(system\_name) \\
        説明入れる
    \item 要求仕様書のディレクトリパス(requirement\_dir) \\
        説明入れる
    \item テストケースのディレクトリパス(testcase\_dir) \\
        説明入れる
    \item 類似度の閾値(similality\_threshold) \\
        説明入れる
    \item 事前条件と前提条件の重み(precondition) \\
        説明入れる
    \item トリガーと入力・操作の重み(trigger\_operation) \\
        説明入れる
    \item システム応答と期待出力の重み(response\_output) \\
        説明入れる
\end{itemize}

\begin{figure}[t]
    \setbox0\vbox{
        \vbox{target\_project:}
        \vbox{\ \ source\_path: target\_project/sample\_project/src}
        \vbox{\ \ test\_exec\_path: target\_project/sample\_project}
        \vbox{\ \ test\_exec\_cmd: bash gtest\_all.sh}
	}
	\centerline{\fbox{\box0}}
    \caption{vitt.yamlの記述例}
    \label{fig:setting_sample}
\end{figure}
設定ファイルvitt.yamlの記述例を、図\ref{fig:setting_sample}に示す。

% ツール使用時の流れ
\section{ツール使用時の流れ}