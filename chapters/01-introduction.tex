\chapter{はじめに}\label{cha:Introduction}

近年、情報技術は会社の業務や人々の生活など、広範囲にわたって使われるようになった。これに伴って、システムの需要の増大および顧客要求の多様化による、システムの多機能化、
大規模化、複雑化が進んでいる [1]。その一方で、ソフトウェアの不具合によるシステム障害がもたらす経済的、社会的影響は大きな問題となっている [2]。
このことから、ソフトウェアの品質確保がより重要視されるようになった。ソフトウェアの品質向上のための手法の 1 つとして、ソフトウェアテストがある。テストの最終目標が「品
質評価」を行うことであると考えるならば、ソフトウェアテストは、計画、設計、準備、実施、および、品質評価という 5 つの段階で構成される [3]。
ソフトウェアテストの活動において、準備の段階で用意するものの 1 つにテストケースがある。テストケースとは、入力値、実行事前条件、期待結果、そして、実行事後条件の組み
合わせであり、プログラムの特定の実行パスを用いることや指定された要件の遵守を検証することといった、特定の目的またはテスト条件のために記述される [4]。
テストケースの設計および作成は、人手で行われることがほとんどである。そのため、テストケースの設計および作成段階において、テストケースに抜けや不備が含まれてしまう可
能性がある [3]。テストケースに抜けや不備が含まれていた場合、ソフトウェアテストを行ってもプログラムの不具合を発見できない可能性が高くなるため、プログラムに不具合が含ま
れたまま、ソフトウェアの開発が終了してしまう。発見されなかったプログラムの不具合は、稼働後のシステム障害につながる。このことから、テストケースに抜けや不備が含まれてい
た場合、ソフトウェアの信頼性が低下することになる。
