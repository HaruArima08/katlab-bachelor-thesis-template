\chapter{\tool{}の実装}\label{cha:Implementation}

本章では、試作した \tool の実装について説明する。
\tool の構造を、図\ref{fig:system}に示す。

\begin{figure}[tp]
    \begin{center}
            \includegraphics[width=18cm]{images/system.png}
            \caption{\tool の構造}
            \label{fig:system}
    \end{center}
\end{figure}

\tool は、以下の$3$つの処理部から構成する。
\begin{itemize}
    \item 入力情報取得部
    \item 対応関係分析部
    \item 出力図生成部部
\end{itemize}
以降、$3$つの処理部についてそれぞれ説明する。

% 入力情報取得部
\section{入力情報取得部}\label{sec:input-data-get-section}
入力情報取得部

\subsection{処理}

\subsection{処理}

% 対応関係分析部
\section{対応関係分析部}

\subsection{処理}

\subsection{処理}

\subsection{処理}

% 出力図生成部
\section{出力図生成部}

\subsection{処理}

\subsection{処理}

\subsection{処理}

