\chapter{\tool{}の実装}\label{cha:Implementation}

本章では、試作した \tool の実装について説明する。
\tool の構造を、図\ref{fig:system}に示す。

\begin{figure}[tp]
    \begin{center}
            \includegraphics[width=18cm]{images/system.png}
            \caption{\tool の構造}
            \label{fig:system}
    \end{center}
\end{figure}

\tool は、以下の$3$つの処理部から構成する。
\begin{itemize}
    \item 入力解析部
    \item 対応関係分析部
    \item 出力図生成部
\end{itemize}
以降、$3$つの処理部についてそれぞれ説明する。

% 入力解析部
\section{入力解析部}\label{sec:input-data-analysis}
入力解析部は、入力として要求仕様書とテストケースのcsvファイルを受け取り、
要求仕様情報リストとテストケース情報リストを出力とする。出力は、対応関係分析部(\ref{sec:correspondence-analysis}節で後述)と出力図生成部(\ref{sec:diagram-generation}節で後述)に渡す。

入力解析部では、以下の$2$つの処理を行う。
\begin{itemize}
    \item 要求仕様情報抽出処理
    \item テストケース情報抽出処理
\end{itemize}
以降、それぞれの処理について説明する。

\subsection{要求仕様情報抽出処理}\label{sec:req-spec-data-extraction}
要求仕様情報抽出は、要求仕様書のcsvファイルを読み込み、要求仕様情報リストを生成する処理である。

\subsection{テストケース情報抽出処理}\label{sec:testcase-data-extraction}

% 対応関係分析部
\section{対応関係分析部}\label{sec:correspondence-analysis}

\subsection{処理}

\subsection{処理}

\subsection{処理}

% 出力図生成部
\section{出力図生成部}\label{sec:diagram-generation}

\subsection{処理}

\subsection{処理}

\subsection{処理}

